\chapter{Conclusiones y perspectivas}

En el presente trabajo se realizó un estudio de una mezcla herbicida 2,4-D/nanotubo de carbono/agua mediante simulación de dinámica molecular. Los resultados que se obtuvieron de función de distribución radial, número de coordinación, desplazamiento cuadrático medio y coeficiente de difusión en función de la temperatura y presión fija, muestran que las moléculas 2,4-D son adsorbidas por el nanotubo de carbono (6,5). De acuerdo con las gráficas obtenidas en la mezcla se puede observar que el agua mantiene su estructura similar a la que se presenta en el agua pura. Adicionalmente se determinó que cerca del nanotubo se pueden encontrar en de 3 a 7 moléculas 2,4-D dependiendo de la temperatura del sistema. También se encontró que el número de moléculas y su cercanía al nanotubo crecen conforme la temperatura aumenta.\\

Adicionalmente, el coeficiente de difusión del nanotubo y herbicida se comportan de manera similar, ambos fluctúan a medida que la temperatura aumenta. Como conclusión general de esta tesis, todos nuestros resultados sugieren que los nanotubos de carbono (6,5) de pared simple atraen moléculas de herbicida 2,4-D en un rango amplio de temperaturas, pueden actuar como adsorbedores de este herbicida, y por tanto, podrían ser útiles en la remoción de contaminantes y el saneamiento de agua o en aplicaciones específicas.\\

Una conclusión importante al respecto de los elementos usados en al simulación molecular de la mezcla, es que el campo de fuerzas empleado para describir las interacciones es adecuado para capturar la fenomenología experimental observada en \cite{rocha2017}, hecho que al inicio de esta investigación era desconocido.\\

Basado en los resultados aquí presentados se pueden plantear diferentes perspectivas. Entre ella se podría extender la investigación a la simulación y modelado de aplicación de fuerzas sobre el nanotubo de carbono para emular múltiples métodos de remoción de nanotubos de carbono del agua. También, se podría explorar la simulación con nanotubos de carbono mas largos y con múltiples moléculas como por ejemplo, las presentadas en el estudio de \cite{rocha2017}.