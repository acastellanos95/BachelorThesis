\chapter{Propiedades macroscópicas y de estructura}
\section{Función de distribución radial}
Usando la integral de configuración clásica en la ecuación \ref{funcpartclasconfig}\\

\begin{equation} \label{intconfclas}
    Z_N = \int e^{-\beta \mathcal{V}(x_1,...,z_N)}dx_1dy_1...dy_N dz_N
\end{equation}\\

La densidad de probabilidad para la partícula 1 en $\mathbf{r}_1$, partícula 2 en $\mathbf{r}_2$,..., es \cite{feynman1972statistical}:\\

\begin{equation}
    \frac{e^{-\beta \mathcal{V}(x_1,...,z_N)}}{\int e^{-\beta \mathcal{V}(x_1,...,z_N)}dx_1...dz_N}
\end{equation}\\

La densidad de probabilidad de encontrar una partícula en $\mathbf{r}_1$ y otra en $\mathbf{r}_2$ esta definida como \cite{feynman1972statistical}:\\

\begin{equation}
    g(r_{12}) = \frac{N(N-1)}{Z_N}
    \int e^{-\beta\mathcal{V}}dx_3...dz_N
\end{equation}\\

$g(r_{12})$ es la \textbf{Función de distribución radial}\\

La interpretación física de la función de distribucion radial es para una molécula fija en $\mathbf{r}_1$ es posible encontrar otros número de moléculas en $\mathbf{r}_2$.

\section{Propiedades macroscópicas del sistema}

\subsection{Energía}

El promedio de la energía es trivialmente calculada como el promedio de la ecuación (\ref{hamiltoniano}) \cite{Allen2017}:

\begin{equation} \label{promenergia}
    \langle E \rangle = \left \langle\sum_{i=1}^{N} \frac{1}{2 m_i}\dot{\vec{p_i}}^2 \right \rangle+ \left \langle\sum_{i=1}^{N-1}\sum_{j=i+1}^{N} u(r_{ij})\right \rangle
\end{equation}

\subsection{Temperatura}

Por el teorema del virial tenemos la siguiente ecuación \cite{Allen2017}:

\begin{equation} \label{virialtheorem}
    \left \langle p_k \frac{\partial H}{\partial p_k} \right \rangle = k_B T
\end{equation}

Para el hamiltoniano en la ecuación (\ref{hamiltoniano}) encontramos:

\begin{equation} \label{virialtemp}
    k_B T = \left \langle p_k \frac{p_k}{m_k} \right \rangle
\end{equation}

entonces derivamos la siguiente ecuación al incorporar la suma para los 3N términos de N partículas:

\begin{equation} \label{virialsumtemp}
    3Nk_B T = \left \langle \sum_{k=1}^{N}\frac{|p_k|^2}{m_k} \right \rangle
\end{equation}

Para un sistema simulado donde hay $N_C$ constricciones de algún tipo (e.g. descripciones de centro de masa, ángulos rígidos, enlaces rígidos entre otros), la ecuación anterior se convierte en:

\begin{equation} \label{virialsumconsttemp}
    T= \frac{1}{3(N - N_C)k_B}\left \langle \sum_{k=1}^{N}\frac{|p_k|^2}{m_k} \right \rangle
\end{equation}

\subsection{Presión}

La otra ecuación del teorema del virial nos da la siguiente ecuación:

\begin{equation} \label{virialtheoremq}
    \left \langle q_k \frac{\partial H}{\partial q_k} \right \rangle = k_B T
\end{equation}

lo cual nos lleva a:

\begin{equation} \label{virialtheorempress}
    \frac{1}{3}\left \langle \sum_{i=1}^N \mathbf{r_i} \cdot f^{tot}_i \right \rangle = -N k_B T
\end{equation}

si dividimos la fuerza total entre externa e intermolecular llegamos a otra ecuación \cite{Allen2017}:

\begin{equation} \label{virialsumconstpress}
    P = \frac{N}{\left \langle V \right \rangle} K_B T + \left \langle \mathcal{W} \right \rangle
\end{equation}

\begin{equation*}
     \text{donde } \mathcal{W} = \frac{1}{3} \sum_{i=1}^N \mathbf{r_i} \cdot f_i \text{ con $f_i$ fuerzas intermoleculares dado por el campo de fuerzas}
\end{equation*}


