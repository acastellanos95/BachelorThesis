% formato-detallado.tex
% Esqueleto para escribir tesis en la FCFM.
% Actualización del original fcfmskel.tex
% --- ATENCIÓN--
% Hay varias instrucciones comentadas. Es necesario quitar el
% comentario a alguna de ellas y comentar la instrucción de arriba.
% Se marcan con %----

\documentclass[fisica]{fcfmtesis}      %----tesis de fisica
% Opciones     ^^^^^^  posibles son:
% fisica (Lic. Física), fisapl (Lic. Física Aplicada)
% pfa1 (M.C. Ciencias, Física Aplicada), pfa2 (D.C. Ciencias, Física Aplicada)
% matematicas (Lic. Matemáticas), lma (Lic. Mat. Aplicadas), actuaria (Lic. Actuaría)
% mem (Maestría en Educacuón Matemática)
%\include{parche}  %----para tener tres vocales, define \vocalc


\usepackage[T1]{fontenc}		   %%% fonts y unicode UTF8
\usepackage[utf8]{inputenc}        
\usepackage{amsmath,amsfonts,amssymb,amsthm}


% Cualquier paquete necesario debe ser incluido aquí.
% Paquetes ya cargados por default en fcfmtesis:
% babel, latexsym, fancyhdr, graphicx y array

\author{André Fabián Castellanos Aldama}                  % Nombre del autor
\title{Dinámica molecular de nanotubos en agua}         % Título de la tesis
\asesor{Minerva González Melchor}               % Nombre del asesor
\presidente{Horoscopista Fernando} % Nombre del presidente del jurado
\secretario{Homeópata Pérez}       % Nombre del secretario del jurado
\vocala{Cienciologo Fdez.}         % Nombre del 1er vocal del jurado
\vocalb{C. Soto}                   % Nombre del 2do vocal del jurado
%\vocalc{}                         % Nombre del 3er vocal del jurado
                                   % Usarse sólo si se hizo \include{parche}

\date{\today}                      % Fecha del examen profesional.
                                   % Escribirse en el formato mes año:
                                   % por ejemplo \date{Julio de 2018}

\hombre
%\mujer       %----Es una mujer quien escribe su trabajo
%tesina      %----En caso de escribir una tesina.

\begin{document}
\frontmatter
           %\portada         %----quitar comentario en la versión final
           \maketitle       %----quitar comentario en la versión final
           \makeacta        %----quitar comentario en la versión final
           \setcounter{page}{4}		% La primer página con número impreso
           							% debe ser la siguiente al acta (pág. impar)
           							% pero el conteo se hace a partir del
           							% maketitle, (segunda portada)	
          							% El valor 4 puede cambiarse según corresponda.
           \tableofcontents %----quitar comentario en la versión final
           \listoffigures
           \listoftables
\chapter{Agradecimientos}

% Gracias a la doctora Minerva González Melchor por las incontables y valiosas enseñanzas además de su apoyo. Por asesorarme e instruirme en el verano de la investigación, en la práctica profesional y el congreso nacional de física 2020, así como la paciencia en la dirección de esta tesis.\\

% \textcolor{red}{A los miembros del jurado. }\\

% A la doctora Lorena Romero Salazar por las revisiones y el apoyo en el desarrollo de este trabajo. A Isidro Romero Medina por inspirarme a estudiar esta carrera y llevar mis aspiraciones a la computación.\\

% A Rubén Octavio Vélez Salazar por su apoyo y los muchos libros prestados que me encantan leer y hojear.\\

% Al laboratorio nacional de supercómputo del sureste de México por permitirme desarrollar las simulaciones en sus instalaciones a través del proyecto 201901019C "Simulación molecular de fluidos"\ como colaborador de la directora de tesis.\\

% Al cuerpo académico de física computacional de la materia condensada por la beca que me otorgaron.\\

% Al centro de cómputo del instituto de física de la benemérita universidad autónoma de puebla por los recursos computacionales otorgados para este trabajo.\\

% \textcolor{red}{A mis padres, mis abuelos, mi novia y mis amigos.\\}

%---------------TODO------------
%La introducción deberá contener
%a) motivación principal para estudiar los sistemas estudiando en la tesis
%b) algunos antecedentes de trabajos experimentales para la captacion o arrastre de herbicidas, contaminantes o medicamentos usando nanotubos de carbono y alguna descripcion general sobre nanotubos de carbono y herbicidas o contaminantes
%c) el estudio de estos sistemas desde el punto de vista de la simulación molecular y algunas referencias al respecto
%investigar funcionalizacion de los nanotubos de carbono para tener una respuesta.
% Buscar mas textos e informacion sobre los nanotubos de carbono
%Ideas para presentación. Bien citada todas mis ideas, platicar sobre saneamiento de agua y usos de nanotubos de carbono. 
\chapter{Introducción}

El saneamiento del agua es un problema que incrementa junto con la población: en 2019 se usó alrededor de 2900$km^3$ de agua para agricultura \cite{un2020} a partir de esto, nuevos métodos y filosofías al respecto han derivado en múltiples innovaciones para atacar el problema como los siguientes: prevención de la contaminación de ríos; plantas de tratamiento de agua de bajo costo y demás.\\ 

El incremento de la población provoca un incremento de cultivos para alimentación lo cual a su vez genera un crecimiento en el uso de herbicidas o químicos para maximizar la cosecha. Estos químicos usados terminan en el agua subterránea y superficial. En particular, en el río La Laja en Guanajuato se midió en el 2019 con una concentración máxima de 11.309 $\mu g / g$ en suelo y $\mu g / L$ en agua el herbicida 2,4D\cite{ineec2019}.\\

Las técnicas de remoción de herbicidas del agua son de mucho interés, mucha investigación existe en este campo como por ejemplo:

\begin{itemize}
    \item Remoción del herbicida 2,4D usando chitosan, polisacárido no tóxico y biodegradable con excelentes resultados\cite{Nunes2019}.
    \item Remoción de terbutizalina usando carbón activado mediante carbonización hidrotermal con buenos resultados\cite{tasca2019}.
    \item Remoción de varios herbicidas usando hongos de tierras contaminadas con herbicidas\cite{Bordjiba2001}.
\end{itemize}

Los nanotubos de carbono son estructuras tubulares provenientes de láminas de grafeno. Poseen una geometría característica y propiedades semiconductoras así como conductoras dependiendo de su geometría. Esto las ha hecho un objeto de estudio muy recurrente.

En los últimos años se ha estudiado mucho sobre las posibles aplicaciones de los nanotubos de carbono en muchas áreas de interés:

\begin{itemize}
    \item Por sus propiedades semiconductoras se creó un microprocesador de 32 bits usando nanotubos de carbono por primera vez \cite{Hills2019}.
    \item Membranas de nanotubos de carbono como candidatos para filtrar agua \cite{IHSANULLAH2019307}.
    \item En el area de la medicina se ha planteado su uso como nanocápsulas transportadoras de medicamentos \cite{hilder2008}.
\end{itemize}

Algunas de estas aplicaciones son perfectas para hacer uso de la simulación molecular por la complejidad que representan estos sistemas. Se han conducido múltiples simulaciones de nanotubos de carbono en la actualidad sobre estas aplicaciones pero por ser un material moderno no se han conducido simulaciones para muchos otros sistemas en los que podría aplicarse. Tal es el caso de la simulación de remoción de herbicidas en el agua usando nanotubos de carbono. En esta tesis se presentan simulaciones de dinámica molecular de un sistema nanotubo de carbono + agua + herbicida 2,4D para verificar la posible remoción de este herbicida usando nanotubos de carbono ya planteado en un artículo \cite{rocha2017}.

\mainmatter  % A partir de aquí va el trabajo propio, se hará colocando
             % comandos \include{archivo} donde archivo es el nombre de 
             % un archivo de latex con nombre "archivo.tex"
             % Notar que en sistemas Unix las mayúsculas son diferentes
             % a las minúsculas.
%---------TODO----------
% Crear una figura mejor
% nanotubos de carbono, dinámica molecular ademas (ensamble nose-hoover y parrinello-rahman), LINCS
\chapter{Mecánica Estadística de Equilibrio}
\section{Fundamentos teóricos}
La mecánica estadística estudia propiedades de los sistemas macroscópicos termodinámicos a partir de sus propiedades microscópicas. Una cantidades macroscópica de estado (ó macroestado $(U,V,N)$) en un sistema, es el promedio de propiedades microscópicas. Un estado microscópico del sistema es un punto en el espacio fase $(q,p)$ correspondiente al macroestado $(U,V,N)$ este se conoce como microestado y el total de microestados asociados a un macroestado se denota por $\Omega(U,V,N)$.\\

Sin embargo, resolver la dinámica del sistema por cualquiera de los métodos en dinámica (Newton, Lagrange o Hamilton) seria un cálculo imposible y largo. Dos postulados fundamentales de la física estadística crean dicha conexión entre ambas escalas \cite{tuckerman2010}: \\

\begin{itemize}
    \item Postulado de equiprobabilidad: En un sistema aislado todos sus microestados accesibles son igualmente probables, i.e., a un macroestado de equilibrio le corresponden mayor número de microestados.\\
    
    \item Postulado de la entropía: En un sistema en equilibrio, su entropía esta dada por:
    \begin{equation}
        S=Kln(\Omega)
    \end{equation}
    con K una constante y $\Omega$ el numero de microestados correspondientes.
\end{itemize}


\section{Ensambles}

Un ensamble es un conjunto hipotético de $N$ replicas de un sistema que se encuentra en un macroestado dado y cada replica esta en algún microestado accesible.\\

Los ensambles pueden ser definidos dependiendo de las situaciones termodinámicas que se impongan al sistema, y dependiendo de estos se pueden extraer propiedades estáticas macroscópicas como la energía, temperatura, presión, etc. Los ensambles que cumplen esta propiedad estática aun cuando el sistema se encuentra evolucionando en el tiempo se les llaman: \textit{Ensambles de equilibrio}.\\

En teoría clásica de ensambles todas las observables macroscópicas de un sistema están conectados a una función microscópica. Por el primer postulado, se puede calcular el promedio temporal de una variable dinámica $A$ como en la ecuacion \ref{promediotemp}, o por un promedio de ensamble asociado a su probabilidad \ref{promedioequprob} \cite{tuckerman2010}. \\

\begin{equation} \label{promediotemp}
    \langle A\rangle = \lim_{t\to\infty}\frac{1}{t}\int_0^t A(t)dt
\end{equation}\\

\begin{equation} \label{promedioequprob}
    \langle A\rangle =\sum_r A_r P_r
\end{equation}\\

con $P_r$ la probabilidad del r-ésimo microestado y $A_r$ el valor de $A$ en el r-ésimo microestado.\\

\begin{table}[h!]
    \centering
    \begin{tabular}{ |p{1cm}||p{4cm}|  }
    \hline
    \multicolumn{2}{|c|}{Ensambles} \\
    \hline
    NVE   & Microcanónico \\
    NVT   & Canónico \\
    $\mu$VT& Gran Canónico \\
    NPT   & Isotérmico-Isobárico \\
    \hline
    \end{tabular}
    \caption{Tipos de Ensambles}
    \label{tiposEnsamble}
\end{table}

\subsection{Ensamble Canónico NVT}

Este ensamble esta formado por $\mathcal{N}$ copias de un sistema en equilibrio con una fuente de calor a temperatura T. Imaginemos que los $\mathcal{N}$ sistemas del ensamble están en contacto entre si mediante paredes diatérmicas, i.e., para cada sistema en este ensamble, el resto es su fuente de calor como se muestra en la figura \ref{fig:CanonicEns}.\\

\begin{figure}[!h]
    \centering
    \includegraphics[width=.55\textwidth,keepaspectratio=true]{EnsCanonico.png}
    \caption{Representación del ensamble canónico NVT \cite{belof2013alternative}}
    \label{fig:CanonicEns}
\end{figure}

\newpage

Las partes del ensamble se encuentran en alguno de los $E_r$ niveles de energía, estos a su vez toman y ceden energía con su alrededor, es decir, los otro microestados. Notar que puede haber degeneración de algunos microestados.\\

Tenemos:\\
\begin{center}
    $N_1$ sistemas con energía $E_1$\\
    $N_2$ sistemas con energía $E_2$\\
    ...\\
    $N_r$ sistemas con energía $E_r$\\
    ...\\
\end{center}


Cumple:

\begin{equation} \label{restrprob}
    1 = \sum_r P_r \\
\end{equation}

\begin{equation} \label{energiaprob}
    E = U = \langle E\rangle = \sum_r P_r E_r
\end{equation}

\begin{center}
    con $P_r = \frac{N_r}{\mathcal{N}}$
\end{center}


El número de formas de microestados correspondientes en que se presenta esta distribución es:\\

\begin{equation}
    \Omega = \frac{\mathcal{N}!}{\prod_r N_r}
\end{equation}\\

Y por el postulado de la entropía y de acuerdo a la formula de Stirling:

\begin{equation}  \label{entropiaboltz}
    S = kln(\frac{\mathcal{N}}{\prod_r N_r}) = -K\mathcal{N}\sum_r P_rln(P_r)
\end{equation}\\

Usando el método de los multiplicadores de Lagrange se encuentra la función de partición canónica $\mathcal{Z}$:

\begin{equation} \label{probcan}
    P_r = \frac{e^{-\beta E_r}}{\mathcal{Z}}
\end{equation}
\begin{equation} \label{funcpartcan}
    \mathcal{Z} = \sum_r e^{-\beta E_r \quad con\ \beta=\frac{1}{KT}
\end{equation}

La entropía dentro de unos de los sistemas del ensamble es \cite{mandl1988statistical}:

\begin{equation}
    S = k\beta U + kln(\mathcal{Z})
\end{equation}

Todas las propiedades termodinámicas se obtienen de la función de partición canónica:

\begin{equation} \label{energcan}
    U=-\frac{\partial}{\partial \beta}ln(\mathcal{Z})
\end{equation}

\begin{equation} \label{ecestacan}
    PV=KTln(\mathcal{Z})
\end{equation}

El potencial termodinámico del ensamble canónico es la energía libre de Helmholtz:

\begin{equation} \label{potHelm}
    F(N,V,T)=-\frac{1}{\beta}ln(\mathcal{Z})=U-TS
\end{equation}

\subsection{Ensamble isotérmico-isobárico NPT}

El ensamble NPT esta formado por $\mathcal{N}$ copias de un sistema en equilibrio con una fuente de calor a temperatura T y una fuente de Volumen. Los $\mathcal{N}$ sistemas de este ensamble estan en contacto entre si por paredes diatérmicas y flexibles.\\

\begin{figure}[!h]
    \centering
    \includegraphics[width=.55\textwidth,keepaspectratio=true]{nptensemble.png}
    \caption{Representación del ensamble NPT \cite{belof2013alternative}}
    \label{fig:NPTEns}
\end{figure}

\newpage

Tenemos en este sistema lo siguiente:\\

\begin{center}
    $N_{1n}$ sistemas con energía $E_{1n}$ y volumen $V_1$\\
    $N_{2n}$ sistemas con energía $E_{2n}$ y volumen $V_2$\\
    ...\\
    $N_{rn}$ sistemas con energía $E_{rn}$ y volumen $V_r$\\
    ...\\
\end{center}

Lo anterior debe cumplir con:

\begin{equation} \label{restrprobNPT}
    1 = \sum_r P_r \\
\end{equation}

\begin{equation} \label{energiaprobNPT}
    U = \langle E\rangle = \sum_{rn} P_{rn} E_{rn}
\end{equation}

\begin{equation} \label{volprobNPT}
    V = \langle E\rangle = \sum_{rn} P_{rn} V_r
\end{equation}

\begin{center}
    con $P_{rn} = \frac{N_{rn}}{\mathcal{N}}$
\end{center}

El número de formas de microestados correspondientes en que se presenta esta distribución es:\\

\begin{equation} \label{distribucionmultnom}
    \Omega = \frac{\mathcal{N}!}{\prod_{rn} N_{rn}}
\end{equation}\\

Y por el postulado de la entropía y de acuerdo a la formula de Stirling:

\begin{equation}  \label{entropiaboltzNPT}
    S = kln(\frac{\mathcal{N}}{\prod_{rn} N_{rn}}) = -K\mathcal{N}\sum_{rn} P_{rn} ln(P_{rn})
\end{equation}\\

Usando el método de los multiplicadores de Lagrange se encuentra la función de partición para el ensamble NPT $\mathcal{Z}$:

\begin{equation} \label{probNPT}
    P_{rn} = \frac{e^{-\beta E_{rn}-\xi V_r}}{\mathcal{Z}}
\end{equation}
\begin{equation} \label{funcpartNPT}
    \mathcal{Z} = \sum_{rn} e^{-\beta E_{rn}-\xi V_r} \quad con\ \beta=\frac{1}{KT} \quad y\ \xi=\frac{P}{KT}
\end{equation}

La entropía de uno de los sistemas del ensamble es \cite{mcquarrie1976}:

\begin{equation}
    S = k\beta U + k\xi V + kln(\mathcal{Z})
\end{equation}

Todas las propiedades termodinámicas se obtienen de la función de partición encontrada:

\begin{equation} \label{energNPT}
    U=-\frac{\partial}{\partial \beta}ln(\mathcal{Z})
\end{equation}

\begin{equation} \label{volNPT}
    V=-\frac{\partial}{\partial \xi}ln(\mathcal{Z})
\end{equation}

\begin{equation} \label{ecestaNPT}
    PV=KTln(\mathcal{Z})
\end{equation}

El potencial termodinámico del ensamble isotérmico-isobárico es la energía libre de Gibbs \cite{mcquarrie1976}:

\begin{equation} \label{potgibbs}
    G(N,P,T)=-\frac{1}{\beta}ln(\mathcal{Z})=U-TS+PV
\end{equation}\\

\section{Mecánica estadística clásica de N partículas interactuantes
}

En las aplicaciones de la mecánica estadística mas comunes están los cálculos que tienen solución exacta, donde las partículas no interactuan en los sistemas. Sin embargo, esto no siempre es posible en sistemas donde las particulas interactuan, por eso es importante

Supongamos un fluido con $\mathcal{N}$ partículas que interactuan a través de un potencial $\mathcal{V}$ y es conservativo. En el formalismo lagrangiano las ecuaciones de movimiento son \cite{torresdelcastillo_2018}:\\

\begin{equation} \label{lagrangeeq}
    \frac{d}{dt}\frac{\partial L}{\partial \dot r_i} - \frac{\partial L}{\partial \dot r_i} = 0
\end{equation}\\

donde $L$ es el lagrangiano del sistema:

\begin{equation} \label{lagrangiano}
    L = \sum_{i=1}^{N} \frac{1}{2}m_i\dot{\vec{r_i}}^2-\mathcal{V}({\vec{r}}_1,{\vec{r}}_2,...,{\vec{r}}_N)
\end{equation}\\

con $\mathcal{V}$ el potencial del sistema, este potencial es el que caracteriza la interacción entre moléculas \cite{torresdelcastillo_2018}:

\begin{equation}
    F_i({\vec{r}}_1,{\vec{r}}_2,...,{\vec{r}}_N) = -\nabla_{r_i}\mathcal{V}({\vec{r}}_1,{\vec{r}}_2,...,{\vec{r}}_N)
\end{equation}\\

De manera práctica y sin entrar en mucho detalle se puede demostrar facilmente que para el lagrangiano $L$ de la ecuación \ref{lagrangiano}, el hamiltoniano es la energía:

\begin{equation} \label{hamiltoniano}
    H(\vec{q},\vec{p}) = E = \sum_{i=1}^{N} \frac{1}{2 m_i}\dot{\vec{p_i}}^2 + \sum_{i,j=1}^{N(N-1)/2} \mathcal{V}(r_{ij})
\end{equation}\\

En el límite termodinámico clásico semi-cuántico las sumas de energías de las funciones de partición se vuelven integrales en el espacio fase. Para un sistema con partículas clásicas independientes e indistinguibles, su función de partición es \cite{mcquarrie1976}:

\begin{equation} \label{funcpartclas}
    \mathcal{Z} = \frac{1}{N!h^{3N}}\int ...\int e^{-\beta H(\vec{q},\vec{p})}dx_1dy_1dz_1dp_{x_1}dp_{y_1}...dy_N dz_Ndp_{x_N}dp_{y_N}dp_{z_N}
\end{equation}\\

Sustituyendo la ecuación \ref{hamiltoniano} en la ecuación \ref{funcpartclas}\cite{feynman1972statistical}:

\begin{equation} \label{funcpartclasconfig}
    \mathcal{Z} = \frac{1}{N!}\left( \frac{2\pi m}{\beta h^2} \right)^{3N/2}Z_N,\quad con\ Z_N = \int e^{-\beta \mathcal{V}}dx_1dy_1dz_1...dx_N dy_N dz_N
\end{equation}\\

donde $Z_N$ es la integral de configuración clásica.

\section{Función de distribución radial}

En la anterior sección presentamos la integral de configuración clásica en la ecuación \ref{funcpartclasconfig}\\

\begin{equation} \label{intconfclas}
    Z_N = \int e^{-\beta \mathcal{V}(x_1,...,z_N)}dx_1dy_1...dy_N dz_N
\end{equation}\\

La densidad de probabilidad para la partícula 1 en $\mathbf{r}_1$, partícula 2 en $\mathbf{r}_2$,..., es \cite{feynman1972statistical}:\\

\begin{equation}
    \frac{e^{-\beta \mathcal{V}(x_1,...,z_N)}}{\int e^{-\beta \mathcal{V}(x_1,...,z_N)}dx_1...dz_N}
\end{equation}\\

La densidad de probabilidad de encontrar una partícula en $\mathbf{r}_1$ y otra en $\mathbf{r}_2$ esta definida como \cite{feynman1972statistical}:\\

\begin{equation}
    g(r_{12}) = \frac{N(N-1)}{Z_N}
    \int e^{-\beta\mathcal{V}}dx_3...dz_N
\end{equation}\\

$g(r_{12})$ es la \textbf{Función de distribución radial}\\

La interpretación física de la función de distribucion radial es para una molécula fija en $\mathbf{r}_1$ es posible encontrar otros números de moléculas en $\mathbf{r}_2$.

\section{Potencial de Lennard-Jones}

El potencial de Lennard-Jones 12-6 es:

\begin{equation} \label{LJ12-6}
    v^{LJ} = 4\epsilon \left[ \left(\frac{\sigma}{r} \right)^{12}-\left(\frac{\sigma}{r} \right)^{6}\right]
\end{equation}\\

$r:= |\vec{r_1}-\vec{r_2}|$\\

$\sigma: $ valor de r donde $v^{LJ}(r)=0$\\

$\epsilon : $profundidad del pozo del potencial\\

Este potencial entre pares modela enlaces débiles de Van der Waals entre gases nobles. Es usado de manera frecuente en los campos de fuerza los cuales se mencionan mas adelante en el texto. Los valores de $\sigma$ y $\epsilon$ se ajustan a propiedades conocidas del gas en cuestión. A continuacion en la figura \ref{fig:LJ126} se muestra un potencial Lennard-Jones 12-6:\\

\begin{figure}[!h]
    \centering
    \includegraphics[width=1\textwidth,keepaspectratio=true]{LJ.png}
    \caption{Potencial Lennard-Jones}
    \label{fig:LJ126}
\end{figure}

\newpage

Este potencial esta dividido en tres intervalos en el eje $r$ \cite{ADAMS2001763}:

\begin{itemize}
    \item El primer intervalo es el potencial repulsivo que esta antes de la linea roja en la figura \ref{fig:LJ126}, este es para evitar el traslape de nubes electrónicas (Principio de exclusión de pauli).
    \item El pozo del potencial es debido a la cohesión en la fase condensada de la materia.
    \item La parte atractiva de este potencial es creada por dipolos temporales por fluctuaciones de las nubes electronicas \cite{INOUE2011157}.
\end{itemize}

\begingroup
\let\clearpage\relax
\chapter{Nanotubos de carbono}
\endgroup

\section{Estructura de los nanotubos de carbono}

En la simulación presentada se usó un nanotubo de carbono de una capa (SWCNT por sus siglas en ingles), así, es de interés explicar caracteristicas, geometría y algunas propiedades.\\

\begin{table}[h!]
    \centering
    \begin{tabular}{ |p{2cm}|p{3cm}|  }
    \hline
    \multicolumn{2}{|c|}{Caracteristicas} \\
    \hline
    $a_{cc}$   & 1.42 \AA \\
    \hline
    C-C-C($\theta$)   & 120 $\deg$ \\
    \hline
    q & 0e \\
    \hline
    m   & 12.0107 u \\
    \hline
    \end{tabular}
    \caption{Caracteristicas del carbono y enlaces de carbono \cite{Melendez2016}}
    \label{carbono}
\end{table}

\newpage

Los nanotubos de carbonos son una sabana de un arreglo hexagonal de carbono enrollados en un eje para formar un cilindro, fueron descubiertos por Iijima en experimentos usando el método de arco de descarga en 1991 \cite{Iijima1991}.\\

\section{Geometría y notación (n,m)}

\begin{figure}[!h] 
    \centering
    \begin{minipage}{0.45\textwidth}
    \centering
    \includegraphics[width=0.95\textwidth]{ChCNT.png}
    \end{minipage}
    \begin{minipage}{0.45\textwidth}
    \centering
    \includegraphics[width=1.2\textwidth]{NT.png}
    \end{minipage}
    \caption{Chiralidad y simetría en nanotubos \cite{ShigeoChiral}}
\end{figure} \label{fig:GeoVectChir}

La geometría se puede describir usando los vectores $\vec{a_1}$ y $\vec{a_2}$ por convención estan arreglados como se muestra en las figuras [2.1]. Usando estos podemos describir las siguientes caracteristicas \cite{Melendez2016}:\\

\begin{itemize}
    \item Vector chiral (n,m): $C_h = n\vec{a_1} + m\vec{a_2}$
    \item Magnitud del vector chiral: $\left|C_h\right| = \sqrt{3}a_{cc}\sqrt{n^2+m^2+nm}$
    \item Diametro del nanotubo: $d_T = \left|C_h\right|/\pi$
\end{itemize}

El vector de traslación $\vec{T}$ es el vector mas corto perpendicular a $C_h$ que empieza y termina en un punto del arreglo hexagonal:

\begin{itemize}
    \item Vector de traslación: $\vec{T} = \left[\left(2m+n\right)\vec{a_1} - \left(2n+m\right)\vec{a_2}\right]/d_R$
    \item Magnitud del vector de traslación: $\left|\vec{T}\right|=\sqrt{3}a_{cc}\frac{\left|C_h\right|}{d_R}$
\end{itemize}

donde:

\begin{equation}\label{dR}
    d_R =
    \begin{cases} 
    d,& \text{si } n-m \text{ no es múltiplo de } 3d\\
    3d,& \text{si } n-m \text{ es múltiplo de } 3d
    \end{cases}
\end{equation}\\

Los nanotubos estan categorizados por su notación (n,m) \cite{Melendez2016}:

\begin{itemize}
    \item Armchair: (n,n) $\left|C_h\right|=3na_{cc}$ y $\left|\vec{T}\right|=\sqrt{3}a_{cc}$
    \item Zigzag: (n,0) $\left|C_h\right|=\sqrt{3}na_{cc}$ y $\left|\vec{T}\right|=3a_{cc}$
    \item Chiral: (n,m) where 0 < m < n
\end{itemize}

Para un SWCNT (n,m), si n = m el nanotubo es metálico; si n - m es múltiplo de 3, el nanotubo es casi-metálico, de otra manera el nanotubo es un semiconductor. En la figura [2.1] los puntos rojos representan los nanotubos metálicos y los círculos negros representan los semiconductores.\\

Lo anterior asegura que la notación (n,m) define completamente la estructura de un SWCNT.

\begingroup
\let\clearpage\relax
\chapter{Dinámica molecular}
\endgroup

\chapter{Propiedades calculadas}

En este capítulo se describen las ecuaciones y los métodos usados para extraer resultados del sistema estudiado. En particular describimos la función de distribución radial, el número de coordinación, coeficiente de difusión,  la temperatura, energía y presión. Al final de este capítulo se detalla el sistema simulado.

\section{Función de distribución radial y número de coordinación}
% \textcolor{red}{usar mejor la definición de VMD para enlazar bien el marco teórico con la metodología}

Esta función es importante ya que proporciona información estructural promedio del sistema.\\

Esta función da la probabilidad de encontrar un par de átomos a distancia r, relativa a la probabilidad de una distribución completamente aleatoria a la misma densidad. En la dinámica molecular, la función de distribución radial se calcula usando la relación (\ref{gr})\\

\begin{equation} \label{gr}
    g(r)=\lim_{dr\to 0} \frac{\left<N(r, r+dr)\right>}{\frac{4 \pi}{3} \rho \left[(r+dr)^3 - r^3\right]}
\end{equation}\\

\noindent donde $r$ es la distancia entre átomos, $\left<N(r, r+dr)\right>$ es el número promedio de átomos encontrados a una distancia entre $r$ y $r + dr$, $\rho$ es la densidad total del sistema.

El promedio $\left<N(r, r+dr)\right>$ es calculado sobre toda la trayectoria de una simulación.

\begin{figure}[!h]
    \centering
    \includegraphics[width=.2\textwidth,keepaspectratio=true]{gr.png}
    \caption{Evaluación de la función de distribución radial}
    \label{fig:gr}
\end{figure}

% \begin{equation} \label{gr}
%     g(r)=\lim_{dr\to 0} \frac{p(r)}{4\pi \left(N_{pares}/V\right)r^{2}dr}
% \end{equation}\\

% donde $r$ es la distancia entre átomos, $p(r)$ es el promedio del número de pares de átomos encontrados a una distancia entre $r$ y $r+dr$, V el volumen total del sistema y $N_{pares}$ es el número de pares únicos de átomos donde un átomo es de cada uno de dos conjuntos, $sel_1$ y $sel_2$.\\

% \begin{figure}[!h]
%     \centering
%     \includegraphics[width=.2\textwidth,keepaspectratio=true]{gr.png}
%     \caption{Evaluación de la función de distribución radial}
%     \label{fig:gr}
% \end{figure}

% El promedio $p(r)$ es calculado sobre toda la trayectoria de una simulación con $N_{cuadros}$ el número total de cuadros en la simulación. Este promedio toma la forma de:\\

% \begin{equation}\label{p}
%     p(r)=\frac{1}{N_{cuadros}}\sum^{N_{cuadros}}_i \sum_{j\in sel1} \sum_{k\in sel2} \sum_{\kappa} d_{\kappa}(r_{ijk})
% \end{equation}\\

% donde $\kappa$ son los índices de los contenedores del histograma y

% \begin{equation}
%     d_{\kappa}(r_{ijk}) = 
%     \begin{cases}
%         1 / \Delta r &\text{si $r_\kappa \leqq r<r_\kappa + \Delta r$ y $r_\kappa \leqq r_{ijk}<r_\kappa + \Delta r$}\\
%         0 & \text{de otra manera}\\
%     \end{cases}
% \end{equation}

% donde $\Delta r$ es el ancho de los contenedores del histograma y $r_\kappa$ es la distancia mínima asociada con cada contenedor, dado por:

% \begin{equation}
%     r_\kappa = r_0 + \kappa \Delta r
% \end{equation}

% con $r_0$ el límite inferior del histograma.\\

% El cálculo de la función de distribución radial también cumple con la convención de mínima imagen. El límite superior del histograma debería no ser mayor a $L/2$.\\

El número de coordinación se calcula como

\begin{equation}
    N(r) =  \int_0^{r_m} \rho g(r)4\pir^{2}dr
\end{equation}

\noindent donde $r_m$ es la distancia en que se encuentra el primer mínimo de $g(r)$ y $\rho$ es la densidad de bulto del sistema. Este es un promedio del número de vecinos hasta $r_m$ \cite{GOCHENOUR201841}. Para el caso de la simulación, es el promedio en todos los cuadros.\\

\section{Propiedades macroscópicas del sistema}

\subsection{Energía}

El promedio de la energía se calcula como el promedio de la ecuación (\ref{hamiltoniano}) \cite{Allen2017}:

\begin{equation} \label{promenergia}
    \langle E \rangle = \left \langle\sum_{i=1}^{N} \frac{1}{2 m_i}\dot{\mathbf{p}}_i^2 \right \rangle+ \left \langle\sum_{i=1}^{N-1}\sum_{j=i+1}^{N} u(r_{ij})\right \rangle
\end{equation}

\subsection{Temperatura}

Aunque la temperatura se mantiene fija en la simulación proporcionamos su descripción \cite{Allen2017}.

\begin{equation} \label{virialtheorem}
    \left \langle p_k \frac{\partial H}{\partial p_k} \right \rangle = k_B T.
\end{equation}

Para el hamiltoniano en la ecuación (\ref{hamiltoniano}) encontramos:

\begin{equation} \label{virialtemp}
    k_B T = \left \langle p_k \frac{p_k}{m_k} \right \rangle
\end{equation}

\noindent entonces derivamos la siguiente ecuación al incorporar la suma para los 3N términos de N partículas:

\begin{equation} \label{virialsumtemp}
    3Nk_B T = \left \langle \sum_{k=1}^{N}\frac{|p_k|^2}{m_k} \right \rangle.
\end{equation}

Para un sistema donde hay $N_C$ restricciones de algún tipo (e.g. descripciones de centro de masa, ángulos rígidos, enlaces rígidos entre otros), la ecuación anterior se convierte en:

\begin{equation} \label{virialsumconsttemp}
    T= \frac{1}{3(N - N_C)k_B}\left \langle \sum_{k=1}^{N}\frac{|p_k|^2}{m_k} \right \rangle
\end{equation}

\subsection{Presión}

% La otra ecuación del teorema del virial nos da la siguiente ecuación:

% \begin{equation} \label{virialtheoremq}
%     \left \langle q_k \frac{\partial H}{\partial q_k} \right \rangle = k_B T
% \end{equation}

% lo cual nos lleva a:

% \begin{equation} \label{virialtheorempress}
%     \frac{1}{3}\left \langle \sum_{i=1}^N \mathbf{r_i} \cdot \mathbf{f}^{tot}_i \right \rangle = -N k_B T
% \end{equation}

% si dividimos la fuerza total entre externa e intermolecular llegamos a otra ecuación \cite{Allen2017}:

La presión promedio se determina considerando su contribución cinética y la parte debida a interacciones, a través de

\begin{equation} \label{virialsumconstpress}
    P = \frac{N}{\left \langle V \right \rangle} k_B T + \frac{\left \langle \mathcal{W} \right \rangle}{\left \langle V \right \rangle}
\end{equation}

\noindent donde $\mathcal{W} = \frac{1}{3} \sum_{i=1}^N \mathbf{r_i} \cdot \mathbf{f}_i$ con $\mathbf{f}_i$ fuerzas intermoleculares las cuales se obtienen a través del campo de fuerzas.

\section{Coeficiente de difusión}

% La ecuación de difusión describe como alguna variable se propaga en un sistema, por ejemplo, unir una barra metálica caliente con otra fría (conocida como la ecuación de difusión de calor).

% La difusión molecular puede interpretar como dos sistemas con diferentes temperaturas o variable termodinámica fluye hacia el equilibrio.

% Por la naturaleza molecular de un sistema, se puede modelar las interacciones de las moléculas como una caminata aleatoria(así ignoramos la interacción o la implicación de momentos, fuerzas, entre otros).

% Podemos hacer primero un modelo de esta caminata aleatoria de manera discreta en el espacio y tiempo hasta llevarlo de manera continua en espacio y tiempo tomando en cuenta el promedio del movimiento de partículas

% Primero describimos el modelo aleatorio en una dimensión. En una sección de recta numérica dejamos como ejemplo mil partículas (con distribución de 50-50 de moverse a ambos lados) distribuidas de igual manera a la mitad del lado izquierdo con una condición de frontera que si una partícula se encuentra en alguna de las frontera, su probabilidad es 50% y 50% en quedarse en el mismo lugar o moverse contrario a la frontera.

%En promedio mitad de la densidad lineal de partículas en un i-esimo delta x se movera hacia la izquierda y la otra mitad a la derecha y la mitad de la densidad lineal de los vecinos entrará

% En ecuación esto es, delta rho_i = 1/2[(rho_{i+1}-rho_{i})-(rho_{i}-rho_{i-1})]

% La constante de difusión da la idea de que tan rápido se mueven las partículas

Los coeficientes de transporte como el coeficiente de difusión describen la relajación de variables dinámicas en la escala macroscópica. Siempre considerando que el tiempo y los límites a gran escala son los adecuados, pueden ser descritos en términos de funciones de correlación de tiempo en equilibrio microscópicos \cite{Allen2017}. Para un sistema en equilibrio, el coeficiente de difusión de las partículas de tipo A, $D_A$, se obtiene a través de \cite{gromacsdoc}

\begin{equation} \label{diffusioncoeff}
    D_A = \frac{1}{6} \lim_{t \to \infty} \left\langle\left\| \mathbf{r}_{i}(t) - \mathbf{r}_{i}(0) \right\|^2 \right\rangle_{i \in A},
\end{equation}

\noindent es decir, el coeficiente de difusión está dado por el desplazamiento cuadrático medio del centro de masa de las posiciones de las moléculas. Este representa la movilidad de las moléculas en la solución, de igual manera que la conductividad eléctrica representa la de los electrones a través del material.\\

En el programa de simulación usado, se puede calcular el coeficiente de difusión de un grupo de moléculas o átomos \cite{gromacsdoc}.

\section{Sistema simulado}

El sistema se simuló en un rango de temperaturas de 280 a 370 K, que consta de 16807 moléculas de agua, 10 moléculas de ácido 2,4-diclorofenoxiacético (2,4-D) y un nanotubo de carbono (6, 5) con una longitud alrededor de 38nm. En todas las simulaciones se usó el termostato de Nosé-Hoover acoplado con un barostato Parrinello-Rahman con $\tau_t$ de 0.4 ps y $\tau_p$ de 1.6 ps, respectivamente.\\

Todas las simulaciones constaron de 20 ns de tiempo de simulación, se usó el algoritmo de salto de rana y radio de corte de 1.2 nm. La validez de los parámetros del termostato y barostato fue verificada revisando que las condiciones de simulación fueron mantenidas de manera adecuada conforme los valores impuestos en la simulación NPT, como se muestra en la tabla \ref{tab:promediostemppres}.

\begin{table}[h!]
    \centering
    \begin{tabular}{ |m{6em}|m{5em}||m{5em}|m{5em}|  }
    \hline
    Temperatura (K) & Error & Presión (bar) & Error \\
    \hline
    \hline
    280.003 & 0.0018 & 0.840079 & 0.12 \\
    298.149 & 0.0014 & 1.11589 & 0.099 \\
    309.998 & 0.0013 & 1.28056 & 0.2 \\
    319.996 & 0.0012 & 1.01285 & 0.12 \\
    329.998 & 0.0016 & 1.12986 & 0.079 \\
    340 & 0.0019 & 1.11812 & 0.13 \\
    349.998 & 0.0028 & 1.30148 & 0.13 \\
    360.003 & 0.0022 & 1.02411 & 0.13 \\
    370.002 & 0.002 & 1.1182 & 0.13 \\
    \hline
    \end{tabular}
    \caption{Promedio de temperatura y presión de los sistemas simulados. El error es calculado por el metodo de promedio de bloques \cite{gromacsdoc}.}
    \label{tab:promediostemppres}
\end{table}


% y el campo de fuerzas usado en el sistema es GROMOS 54A7, que tiene la siguiente forma en nuestras simulaciones:
% \begin{itemize}
%     \item Número de moléculas: 16807 moléculas de agua, 10 moléculas de ácido 2,4-diclorofenoxiacético (2,4-D) y un nanotubo de carbono (6, 5) con 364 átomos.
%     \item Integrador de todos los sistemas: salto de rana para las ecuaciones de movimiento de Newton.
%     \item Paso de tiempo: 0.002 ps, número de pasos: 10000000, Intervalo de tiempo: 20 ns.
%     \item radio de corte para la lista de vecinos de corto alcance: 1.2 nm, radio de corte para interacción de coulomb: 1.2 nm, radio de corte para interacción LJ 12-6: 1.2 nm.
%     \item Corrección de dispersión de largo alcance para energía y presión.
%     \item Acoplamiento de temperatura: ensamble extendido de nose-hoover con $\tau_t$: 0.4 ps y con las temperaturas antes mencionadas.
%     \item Acoplamiento de presión: ensamble extendido de parrinello-rahman con $\tau_p$: 1.6 ps a presión: 1 bar.
%     \item Algoritmo de constricción: LINCS(LINear Constraint Solver).
% \end{itemize}





\chapter{Resultados}
Texto

\section{Sistema simulado}

Los sistemas se simularon a las temperaturas de: 280 K, 298.15 K, 310 K, 320 K, 330 K, 340 K, 350 K, 360 K y 370 K; y con las siguientes caracteristicas:

\begin{itemize}
    \item Número de moléculas: 16807 moléculas de agua, 10 moléculas de ácido 2,4-diclorofenoxiacético (2,4-D) y un nanotubo de carbono (6, 5) con 364 átomos.
    \item Integrador de todos los sistemas: salto de rana para las ecuaciones de movimiento de Newton.
    \item Paso de tiempo: 0.002 ps, número de pasos: 10000000, Intervalo de tiempo: 20 ns.
    \item radio de corte para la lista de vecinos de corto alcance: 1.2 nm, radio de corte para interacción de coulomb: 1.2 nm, radio de corte para interacción LJ 12-6: 1.2 nm.
    \item Corrección de dispersión de largo alcance para energía y presión.
    \item Acoplamiento de temperatura: ensamble extendido de nose-hoover con $\tau_t$: 0.4 ps y con las temperaturas antes mencionadas.
    \item Acoplamiento de presión: ensamble extendido de parrinello-rahman con $\tau_p$: 1.6 ps a presión: 1 bar.
    \item Algoritmo de constricción: LINCS(LINear Constraint Solver).
\end{itemize}

\chapter{Conclusión}

En el presente trabajo se realizó un estudio de una mezcla agua + CNT(6,5) + 10 moléculas 2,4-D mediante simulación de dinámica molecular. Los resultados que se obtuvieron nos permite afirmar que las moleculas 2,4-D son adsorbidas por nanotubos de carbono (6,5). Gracias a las gráficas obtenidas se puede observar que el agua mantiene su estructura, se pueden encontrar en promedio cerca del nanotubo de 3 a 7 moléculas 2,4-D segun la temperatura del sistema. También se encontró que el número de moléculas y la cercanía al nanotubo crecen conforme la temperatura aumenta. Como conclusión, los nanotubos de carbono de pared simple (6,5) adsorben moléculas 2,4-D.

\appendix % A partir de esta línea van los apéndices. Quitar si no habrá apéndices
\chapter{Apéndice A}\label{chapter:apendicea}

\textcolor{red}{Encabezados de tablas}

\section{Parámetros de los modelos}

\subsection{Modelo de agua}

El modelo de agua elegido para estas simulaciones fue el modelo SPC/E(punto simple punto de carga extendido) rígido, tiene cargas situadas en los tres átomos y la interacción Lennard-Jones con otras moléculas esta situada en el oxígeno. La figura (\ref{fig:SPCE}) es una visualización de la molécula y la tabla (\ref{SPCEpar}) son los parámetros del modelo.

\begin{figure}[!h]
    \centering
    \includegraphics[width=.9\textwidth,keepaspectratio=true]{SPCE.png}
    \caption{Molécula de agua}
    \label{fig:SPCE}
\end{figure}

\begin{table}[h!]
    \centering

    \begin{tabular}{ |p{1cm}|p{4cm}|  }
    \hline
    $\sigma$  & 3.166 \AA \\
    $\epsilon$& 0.650 KJ $mol^{-1}$ \\
    $r_{OH}$  & 1.000 \AA \\
    $\angle_{HOH}$&109.47 deg \\
    $q_{O}$   & -0.8476 e \\
    $q_{H}$   & 0.4238 e \\
    \hline
    \end{tabular}
    \caption{Parámetros del modelo SPCE}
    \label{SPCEpar}
\end{table}

\newpage

\subsection{Modelo del sistema solvatado}

El sistema se compone de un nanotubo de carbono (6,5) y 10 moléculas de 2,4D. La siguientes tablas contienen los parámetros de simulación.

\begin{table}[!h]
    \centering
    \begin{tabular}{|c|c|c|c|}
    \hline
    $\epsilon_{cc}$ & 0.4058kJ/mol & $q_{o}$   & -0.8476e \\
    $\epsilon_{oo}$ & 0.6502kJ/mol &  $q_H$    & 0.4238e \\
    $\epsilon_{co}$ & 0.5137kJ/mol & $K^b_{CC}$& 3.2236X$10^6$ $\frac{kJ}{molnm^4}$\\
    $\sigma_{cc}$   & 3.361        & $b_{ij}$  & 1.42 \AA\\
    $\sigma_{oo}$   & 3.166        & $K^{\theta}_{CCC}$& 560 $\frac{kJ}{mol}$ \\
    $\sigma_{co}$   & 3.262        & $\theta^0_{ijk}$  & $120^{\circ}$ \\
    $r_{OH}$        & 1 \AA        & $k^{\phi}_{CCCC}$ & 5.86 $\frac{kJ}{mol}$ \\
    $\theta_{HOH}$  & $109.47^{\circ}$ & $\phi_s$         & $180^{\circ}$ \\
    \hline
    \end{tabular}
    \caption{Parámetros del agua rígida y el nanotubo de carbono. \cite{meng2008}}
    \label{tab:cnth2oparameters}
\end{table}

\begin{figure}[!h]
    \centering
    \includegraphics[width=.9\textwidth,keepaspectratio=true]{figura_nueva_tarea.png}
    \caption{Molécula 2,4-D}
    \label{fig:24Dfigure}
\end{figure}

% \begin{table}[!h]
% \caption{\label{tab:cnth2oparameters}
% Parámetros del agua rígida y el nanotubo de carbono. \cite{meng2008}}
% \centering
% \begin{tabular}{| c | c || c | c |}
% % &$r_c$ (\AA)&
% % &$r_c$ (\AA)\\
% \hline
% \large{$\epsilon_{cc}$}\footnotemark[1] & 0.4058 kJ/mol & {\large q}_O & -0.8476 e \\
% \large{$\epsilon_{oo}$}\footnotemark[1]& 0.6502 kJ/mol &{\large q}_H& 0.4238 e \\
% \large{$\epsilon_{co}$}\footnotemark[1]& 0.5137 kJ/mol &{\large K}^b_{CC}& 3.2236X10^6 $\frac{kJ}{mol\ nm^4}$\\
% \large{$\sigma_{cc}$}\footnotemark[1]&   3.361         &{\large b}_{ij}& 1.42 \AA\\
% \large{$\sigma_{oo}$}\footnotemark[1]&   3.166         &{\large K}^{$\theta$}_{CCC}& $560 \frac{kJ}{mol}$ \\
% \large{$\sigma_{co}$}\footnotemark[1]&   3.262         &{\large $\theta$}^0_{ijk}& $120^{\circ}$ \\
% {\large r}_{OH}& 1 \AA                                 &{\large k}^{$\phi$}_{CCCC}& 5.86 $\frac{kJ}{mol}$ \\
% {\large $\theta$}_{HOH}& 109.47^{\circ}                &{\large $\phi$}_s& $180^{\circ}$ \\
% \hline
% \end{tabular}
% \footnotetext[1]{Provided in GROMACS}
% \end{table}

% [ moleculetype ]
% ; Name   nrexcl
% SAYK     3
% [ atoms ]
% ;  nr  type  resnr  resid  atom  cgnr  charge    mass
%     1 CLAro    1    SAYK    CL1    1   -0.095  35.4530
%     2  CAro    1    SAYK     C4    2   -0.125  12.0110
%     3  CAro    1    SAYK     C3    3    0.070  12.0110
%     4    HC    1    SAYK     H2    4    0.116   1.0080
%     5  CAro    1    SAYK     C2    5   -0.185  12.0110
%     6 CLAro    1    SAYK     CL    6   -0.070  35.4530
%     7  CAro    1    SAYK     C1    7    0.491  12.0110
%     8  CAro    1    SAYK     C6    8   -0.375  12.0110
%     9    HC    1    SAYK     H4    9    0.212   1.0080
%   10  CAro    1    SAYK     C5   10    0.037  12.0110
%   11    HC    1    SAYK     H3   11    0.130   1.0080
%   12    OE    1    SAYK     O3   12   -0.431  15.9994
%   13  CPos    1    SAYK     C7   13    0.038  12.0110
%   14    HC    1    SAYK     H5   14    0.099   1.0080
%   15    HC    1    SAYK     H6   15    0.099   1.0080
%   16  CPos    1    SAYK     C8   16    0.683  12.0110
%   17 OEOpt    1    SAYK     O2   17   -0.549  15.9994
%   18    OA    1    SAYK     O1   18   -0.610  15.9994
%   19  HS14    1    SAYK     H1   19    0.465   1.0080
% ; total charge of the molecule:   0.000
% [ bonds ]
% ;  ai   aj  funct   c0         c1
%     1    2    2   0.1760   1.4366e+06
%     2    3    2   0.1390   8.6600e+06
%     2   10    2   0.1390   8.6600e+06
%     3    4    2   0.1090   1.2300e+07
%     3    5    2   0.1390   8.6600e+06
%     5    6    2   0.1760   1.4366e+06
%     5    7    2   0.1400   8.5400e+06
%     7    8    2   0.1390   8.6600e+06
%     7   12    2   0.1380   1.1000e+07
%     8    9    2   0.1090   1.2300e+07
%     8   10    2   0.1390   8.6600e+06
%   10   11    2   0.1090   1.2300e+07
%   12   13    2   0.1430   8.1800e+06
%   13   14    2   0.1090   1.2300e+07
%   13   15    2   0.1090   1.2300e+07
%   13   16    2   0.1530   7.1500e+06
%   16   17    2   0.1210   1.2977e+07
%   16   18    2   0.1350   1.0300e+07
%   18   19    2   0.0983   9.8314e+06
% [ pairs ]
% ;  ai   aj  funct  ;  all 1-4 pairs but the ones excluded in GROMOS itp
%     1    4    1
%     1    5    1
%     1    8    1
%     1   11    1
%     2    6    1
%     2    9    1
%     3   11    1
%     3   12    1
%     4    6    1
%     4    7    1
%     4   10    1
%     5    9    1
%     5   13    1
%     6    8    1
%     6   12    1
%     7   11    1
%     7   14    1
%     7   15    1
%     7   16    1
%     8   13    1
%     9   11    1
%     9   12    1
%   10   12    1
%   12   17    1
%   12   18    1
%   13   19    1
%   14   17    1
%   14   18    1
%   15   17    1
%   15   18    1
%   17   19    1
% [ angles ]
% ;  ai   aj   ak  funct   angle     fc
%     1    2    3    2    120.00   560.00
%     1    2   10    2    120.00   560.00
%     3    2   10    2    120.00   560.00
%     2    3    4    2    120.00   505.00
%     2    3    5    2    120.00   560.00
%     4    3    5    2    120.00   505.00
%     3    5    6    2    120.00   560.00
%     3    5    7    2    120.00   560.00
%     6    5    7    2    120.00   560.00
%     5    7    8    2    120.00   560.00
%     5    7   12    2    121.00   685.00
%     8    7   12    2    120.00   560.00
%     7    8    9    2    120.00   505.00
%     7    8   10    2    120.00   560.00
%     9    8   10    2    120.00   505.00
%     2   10    8    2    120.00   560.00
%     2   10   11    2    120.00   505.00
%     8   10   11    2    120.00   505.00
%     7   12   13    2    119.00  2211.40
%   12   13   14    2    106.75   503.00
%   12   13   15    2    106.75   503.00
%   12   13   16    2    111.00   530.00
%   14   13   15    2    107.57   484.00
%   14   13   16    2    109.60   450.00
%   15   13   16    2    109.60   450.00
%   13   16   17    2    126.00   640.00
%   13   16   18    2    109.50   520.00
%   17   16   18    2    124.00   730.00
%   16   18   19    2    109.50   450.00
% [ dihedrals ]
% ; GROMOS improper dihedrals
% ;  ai   aj   ak   al  funct   angle     fc
%     7    5    8   12    2      0.00   167.36
%     5    3    6    7    2      0.00   167.36
%     3    2    4    5    2      0.00   167.36
%     2    1    3   10    2      0.00   167.36
%   10    2    8   11    2      0.00   167.36
%     8    7    9   10    2      0.00   167.36
%   16   13   17   18    2      0.00   167.36
% [ dihedrals ]
% ;  ai   aj   ak   al  funct    ph0      cp     mult
%     2    3    5    7    1    180.00    41.80    2
%     3    2   10    8    1    180.00    41.80    2
%     3    5    7    8    1    180.00    41.80    2
%     5    7    8   10    1    180.00    41.80    2
%     5    7   12   13    1      0.00     0.42    2
%     7    8   10    2    1    180.00    41.80    2
%     7   12   13   16    1      0.00     1.26    3
%   10    2    3    5    1    180.00    41.80    2
%   12   13   16   18    1    180.00     1.00    6
%   17   16   18   19    1    180.00     7.11    2
% [ exclusions ]
% ;  ai   aj  funct  ;  GROMOS 1-4 exclusions
%     2    7
%     3    8
%     5   10

\begin{table}[!h]
    \centering
    \begin{tabular}{|c|c|}
    \hline
    \large{$q_{CL}$} & -0.070 e \\
    \large{$q_{CL1}$} & -0.095 e \\
    \large{$q_{C1}$} & 0.491 e \\
    \large{$q_{C2}$} & -0.185 e \\
    \large{$q_{C3}$} & 0.070 e \\
    \large{$q_{C4}$} & -0.125 e \\
    \large{$q_{C5}$} & 0.037 e \\
    \large{$q_{C6}$} & -0.375 e \\
    \large{$q_{C7}$} & 0.038 e \\
    \large{$q_{C8}$} & 0.683 e \\
    \large{$q_{H1}$} & 0.465 e \\
    \large{$q_{H2}$} & 0.116 e \\
    \large{$q_{H3}$} & 0.130 e \\
    \large{$q_{H4}$} & 0.212 e \\
    \large{$q_{H5}$} & 0.099 e \\
    \large{$q_{H6}$} & 0.099 e \\
    \large{$q_{O1}$} & -0.610 e \\
    \large{$q_{O2}$} & -0.549 e \\
    \large{$q_{O3}$} & -0.431 e \\
    \hline
    \end{tabular}
    \caption{Tabla de cargas en 2,4-D}
    \label{tab:cargas24D}
\end{table}


\begin{table}[!h]
    \centering
    \begin{tabular}{|c|c|}
    \hline
    $K^b_{C1-C2} b_{C1-C2}$   & 8.5400X$10^6$ $\frac{kJ}{molnm^4}$ 1.400 \AA\\
    $K^b_{C2-C3} b_{C2-C3}$   & 8.6600X$10^6$ $\frac{kJ}{molnm^4}$ 1.390 \AA\\
    $K^b_{C3-C4} b_{C3-C4}$   & 8.6600X$10^6$ $\frac{kJ}{molnm^4}$ 1.390 \AA\\
    $K^b_{C4-C5} b_{C4-C5}$   & 8.6600X$10^6$ $\frac{kJ}{molnm^4}$ 1.390 \AA\\
    $K^b_{C5-C6} b_{C5-C6}$   & 8.6600X$10^6$ $\frac{kJ}{molnm^4}$ 1.390 \AA\\
    $K^b_{C7-C8} b_{C7-C8}$   & 7.1500X$10^6$ $\frac{kJ}{molnm^4}$ 1.530 \AA\\
    $K^b_{C1-C6} b_{C1-C6}$   & 8.6600X$10^6$ $\frac{kJ}{molnm^4}$ 1.390 \AA\\
    $K^b_{CL1-C4} b_{CL1-C4}$ & 1.4366X$10^6$ $\frac{kJ}{molnm^4}$ 1.760 \AA\\
    $K^b_{C2-CL} b_{C2-CL}$   & 1.4366X$10^6$ $\frac{kJ}{molnm^4}$ 1.760 \AA\\
    $K^b_{C5-H3} b_{C5-H3}$   & 1.2300X$10^7$ $\frac{kJ}{molnm^4}$ 1.090 \AA\\
    $K^b_{C6-H4} b_{C6-H4}$   & 1.2300X$10^7$ $\frac{kJ}{molnm^4}$ 1.090 \AA\\
    $K^b_{C3-H2} b_{C3-H2}$   & 1.2300X$10^7$ $\frac{kJ}{molnm^4}$ 1.090 \AA\\
    $K^b_{C1-O3} b_{C1-O3}$   & 1.1000X$10^7$ $\frac{kJ}{molnm^4}$ 1.380 \AA\\
    $K^b_{O3-C7} b_{O3-C7}$   & 8.1800X$10^6$ $\frac{kJ}{molnm^4}$ 1.430 \AA\\
    $K^b_{H5-C7} b_{H5-C7}$   & 1.2300X$10^7$ $\frac{kJ}{molnm^4}$ 1.090 \AA\\
    $K^b_{H6-C7} b_{H6-C7}$   & 1.2300X$10^7$ $\frac{kJ}{molnm^4}$ 1.090 \AA\\
    $K^b_{O2-C8} b_{O2-C8}$   & 1.2977X$10^7$ $\frac{kJ}{molnm^4}$ 1.210 \AA\\
    $K^b_{O-C8}  b_{O1-C8}$   & 1.0300X$10^7$ $\frac{kJ}{molnm^4}$ 1.210 \AA\\
    $K^b_{O1-H1} b_{O1-H1}$   & 9.8314X$10^6$ $\frac{kJ}{molnm^4}$ 0.983 \AA\\
    \hline
    \end{tabular}
    \caption{Tabla de parametros de enlace de 2,4-D}
    \label{tab:enlace24D}
\end{table}



\begin{table}[!h]
    \centering
    \begin{tabular}{|c|c|}
    \hline
    $K^{\theta}_{C3-C4-CL1} \theta^0_{C3-C4-CL1}$& 560 $\frac{kJ}{mol}$ $120^{\circ}$\\
    $K^{\theta}_{C4-C5-CL1} \theta^0_{C4-C5-CL1}$& 560 $\frac{kJ}{mol}$ $120^{\circ}$\\
    $K^{\theta}_{C3-C4-C5} \theta^0_{C3-C4-C5}$& 560 $\frac{kJ}{mol}$ $120^{\circ}$\\
    $K^{\theta}_{C2-C3-C4} \theta^0_{C2-C3-C4}$& 560 $\frac{kJ}{mol}$ $120^{\circ}$\\
    $K^{\theta}_{C2-C3-CL} \theta^0_{C2-C3-CL}$& 560 $\frac{kJ}{mol}$ $120^{\circ}$\\
    $K^{\theta}_{C1-C2-C3} \theta^0_{C1-C2-C3}$& 560 $\frac{kJ}{mol}$ $120^{\circ}$\\
    $K^{\theta}_{C1-C2-CL} \theta^0_{C1-C2-CL}$& 560 $\frac{kJ}{mol}$ $120^{\circ}$\\
    $K^{\theta}_{C1-C2-C6} \theta^0_{C1-C2-C6}$& 560 $\frac{kJ}{mol}$ $120^{\circ}$\\
    $K^{\theta}_{C1-C6-O3} \theta^0_{C1-C6-O3}$& 560 $\frac{kJ}{mol}$ $120^{\circ}$\\
    $K^{\theta}_{C1-C6-C5} \theta^0_{C1-C6-C5}$& 560 $\frac{kJ}{mol}$ $120^{\circ}$\\
    $K^{\theta}_{C4-C5-C6} \theta^0_{C4-C5-C6}$& 560 $\frac{kJ}{mol}$ $120^{\circ}$\\

    $K^{\theta}_{C3-C4-H2} \theta^0_{C3-C4-H2}$& 505 $\frac{kJ}{mol}$ $120^{\circ}$\\
    $K^{\theta}_{C2-C3-H2} \theta^0_{C2-C3-H2}$& 505 $\frac{kJ}{mol}$ $120^{\circ}$\\
    $K^{\theta}_{C1-C6-H4} \theta^0_{C1-C6-H4}$& 505 $\frac{kJ}{mol}$ $120^{\circ}$\\
    $K^{\theta}_{C5-C6-H4} \theta^0_{C5-C6-H4}$& 505 $\frac{kJ}{mol}$ $120^{\circ}$\\
    $K^{\theta}_{C4-C5-H3} \theta^0_{C4-C5-H3}$& 505 $\frac{kJ}{mol}$ $120^{\circ}$\\
    $K^{\theta}_{C5-C6-H3} \theta^0_{C5-C6-H3}$& 505 $\frac{kJ}{mol}$ $120^{\circ}$\\
    
    $K^{\theta}_{C7-C8-H5} \theta^0_{C7-C8-H5}$& 450 $\frac{kJ}{mol}$ $109.60^{\circ}$\\
    $K^{\theta}_{C7-C8-H6} \theta^0_{C7-C8-H6}$& 450 $\frac{kJ}{mol}$ $109.60^{\circ}$\\
    $K^{\theta}_{C8-O1-H1} \theta^0_{C8-O1-H1}$& 450 $\frac{kJ}{mol}$ $109.50^{\circ}$\\
    $K^{\theta}_{C7-C8-O1} \theta^0_{C7-C8-O1}$& 520 $\frac{kJ}{mol}$ $109.50^{\circ}$\\
    
    $K^{\theta}_{C7-H5-H6} \theta^0_{C7-H5-H6}$& 484 $\frac{kJ}{mol}$ $107.57^{\circ}$\\
    $K^{\theta}_{O3-C7-H5} \theta^0_{O3-C7-H5}$& 503 $\frac{kJ}{mol}$ $106.75^{\circ}$\\
    $K^{\theta}_{O3-C7-H6} \theta^0_{O3-C7-H6}$& 503 $\frac{kJ}{mol}$ $106.75^{\circ}$\\
    $K^{\theta}_{O3-C7-C8} \theta^0_{O3-C7-C8}$& 530 $\frac{kJ}{mol}$ $111^{\circ}$\\
    $K^{\theta}_{C7-C8-O2} \theta^0_{C7-C8-O2}$& 640 $\frac{kJ}{mol}$ $126^{\circ}$\\
    $K^{\theta}_{C1-C2-O3} \theta^0_{C1-C2-O3}$& 685 $\frac{kJ}{mol}$ $121^{\circ}$\\
    $K^{\theta}_{C8-O2-O1} \theta^0_{C8-O2-O1}$& 730 $\frac{kJ}{mol}$ $124^{\circ}$\\
    $K^{\theta}_{C1-C7-O3} \theta^0_{C1-C7-O3}$& 2211.40 $\frac{kJ}{mol}$ $119^{\circ}$\\
    \hline
    \end{tabular}
    \caption{Tabla de parametros de angulos de 2,4-D}
    \label{tab:angulos24D}
\end{table}

\begin{table}[!h]
    \centering
    \begin{tabular}{|c|c|}
    \hline
    $k^{\xi}_{C1-C2-C6-O3}$ $\xi_{0}$ & 167.36 $\frac{kJ}{molrad^2}$ $0^{\circ}$ \\
    $k^{\xi}_{C1-C2-C3-CL}$ $\xi_{0}$ & 167.36 $\frac{kJ}{molrad^2}$ $0^{\circ}$ \\
    $k^{\xi}_{C2-C3-C4-H2}$ $\xi_{0}$ & 167.36 $\frac{kJ}{molrad^2}$ $0^{\circ}$ \\
    $k^{\xi}_{C3-C4-C5-CL1}$ $\xi_{0}$ & 167.36 $\frac{kJ}{molrad^2}$ $0^{\circ}$ \\
    $k^{\xi}_{C4-C5-C6-H3}$ $\xi_{0}$ & 167.36 $\frac{kJ}{molrad^2}$ $0^{\circ}$ \\
    $k^{\xi}_{C1-C5-C6-H4}$ $\xi_{0}$ & 167.36 $\frac{kJ}{molrad^2}$ $0^{\circ}$ \\
    $k^{\xi}_{C7-C8-O1-O2}$ $\xi_{0}$ & 167.36 $\frac{kJ}{molrad^2}$ $0^{\circ}$ \\
    
    $k^{\phi}_{C1-C2-C3-C4}$ $\phi_s$ & 41.80 $\frac{kJ}{mol}$ $180^{\circ}$\\
    $k^{\phi}_{C3-C4-C5-C6}$ $\phi_s$ & 41.80 $\frac{kJ}{mol}$ $180^{\circ}$\\
    $k^{\phi}_{C1-C2-C3-C6}$ $\phi_s$ & 41.80 $\frac{kJ}{mol}$ $180^{\circ}$\\
    $k^{\phi}_{C1-C2-C5-C6}$ $\phi_s$ & 41.80 $\frac{kJ}{mol}$ $180^{\circ}$\\
    $k^{\phi}_{C1-C4-C5-C6}$ $\phi_s$ & 41.80 $\frac{kJ}{mol}$ $180^{\circ}$\\
    $k^{\phi}_{C2-C3-C4-C5}$ $\phi_s$ & 41.80 $\frac{kJ}{mol}$ $180^{\circ}$\\
    $k^{\phi}_{C7-C8-O1-O3}$ $\phi_s$ n & 1.00 $\frac{kJ}{mol}$ $180^{\circ}$ 6\\
    $k^{\phi}_{C8-O1-O2-H1}$ $\phi_s$ n & 7.11 $\frac{kJ}{mol}$ $180^{\circ}$ 2\\
    $k^{\phi}_{C1-C2-O3-C7}$ $\phi_s$ n & 0.42 $\frac{kJ}{mol}$ $0^{\circ}$ 2\\
    $k^{\phi}_{C1-C7-C8-O3}$ $\phi_s$ n & 1.26 $\frac{kJ}{mol}$ $0^{\circ}$ 3\\
    \hline
    \end{tabular}
    \caption{Tabla de parametros de angulos dihedros de 2,4-D}
    \label{tab:angulosdih24D}
\end{table}
%               c6              c12                sigma                 epsilon
% C    CLAro      1  0.003974417     6.381584e-06       0.3421967204782815    0.6188115086273643
% C     CAro      1  0.002340624     4.690642e-06      sigma: 0.3550722828524636, epsilon: 0.29199205084165447
% HC	C	      1	4.450960E-04	2.733060E-07      sigma: 0.2915407359639445, epsilon: 0.18121670327032705
% C	OE	          1	2.300953E-03	2.444200E-06          sigma: 0.31942690878969837, epsilon: 0.541525315871144
% C     CPos      1  0.0021771       2.222e-06         sigma: 0.31730550936068447, epsilon: 0.5332768238073807
% C    OEOpt      1  0.00268896      4.85507e-06       sigma: 0.34895427699812803, epsilon: 0.37231728284041216
% C     HS14      1  0               0           
% C	OA	          1	2.300953E-03	2.444200E-06          sigma: 0.31942690878969837, epsilon: 0.541525315871144

% OA    CLAro     1  0.003907054     3.1592e-06       sigma: 0.3052258307853132, epsilon: 1.2079854835809698
% OW    CLAro     1  0.004202794     4.661256e-06     sigma: 0.3217319135230157, epsilon: 0.9473561099431141
% HS14    CLAro   1  0               0
% CLAro    OEOpt  1  0.004565897     6.27532e-06   sigma: 0.33344045848470744, epsilon: 0.830531965485784
% CLAro     CAro  1  0.003974417     6.062792e-06  sigma: 0.3392864686204393, epsilon: 0.6513496789057335
% CLAro     CPos  1  0.00369675      2.872e-06     sigma: 0.3031987401164613, epsilon: 1.1895857035602366
% OE    CLAro     1  0.003907054     3.231e-06        sigma: 0.30637119042109173, epsilon: 1.181141361723615
% HC    CLAro     1  0.00075578      3.53256e-07      sigma: 0.27857941708920847, epsilon: 0.4042418305704644

% OE     CAro     1  0.002300953     2.3221e-06       sigma: 0.3167103054468882, epsilon: 0.5699996456019336
% OA     CAro     1  0.002300953     2.3221e-06       sigma: 0.3167103054468882, epsilon: 0.5699996456019336
% OW     CAro     1  0.002475121     3.426153e-06     sigma: 0.33383744896554585, epsilon: 0.44701914688580746
% HC     CAro     1  0.000445096     2.59653e-07      sigma: 0.2890612938252534, epsilon: 0.19074538828359391
% HS14     CAro   1  0               0
% OEOpt     CAro  1  0.00268896      4.612535e-06  sigma: 0.34598655471395084, epsilon: 0.39189436403192596
% CAro     CPos   1  0.0021771       2.111e-06      sigma: 0.3146069477063536, epsilon: 0.56131743368072
% CAro     CAro   1  0.002340624     4.456321e-06   sigma: 0.3520525292750613, epsilon: 0.30734549359078933

% HC	OA	      1	4.375520E-04	1.353000E-07      sigma: 0.2600427123263571, epsilon: 0.3537541624242424
% HC	OE	      1	4.375520E-04	1.353000E-07      sigma: 0.2600427123263571, epsilon: 0.3537541624242424
% HC	OW	      1	4.706720E-04	1.996290E-07      sigma: 0.2741053705029336, epsilon: 0.2774297967529768
% HC     HS14     1  8.464e-05       1.5129e-08       sigma: 0.23734081428097492, epsilon: 0.1183807521977659
% HC    OEOpt     1  0.000511336     2.68755e-07      sigma: 0.28408067888820254, epsilon: 0.24321827026101842
% HC     CPos     1  0.000414        1.23e-07         sigma: 0.2583156954670584, epsilon: 0.34836585365853656
    
% OE     HS14     1  0               0     
% OE    OEOpt     1  0.002643385     2.4035e-06       sigma: 0.31125329012564956, epsilon: 0.7268030224906386
% OE     CPos     1  0.0021402       1.1e-06          sigma: 0.28302386291723314, epsilon: 1.0410127363636363
    
% OA     CPos     1  0.0021402       1.1e-06          sigma: 0.28302386291723314, epsilon: 1.0410127363636363
% OW     CPos     1  0.0023022       1.623e-06        sigma: 0.2983293017984591, epsilon: 0.8164086321626617
% HS14     CPos   1  0               0
% OEOpt     CPos  1  0.0025011       2.185e-06     sigma: 0.3091861733013462, epsilon: 0.7157324038901602

% OA    OEOpt     1  0.002643385     2.4035e-06       sigma: 0.31125329012564956, epsilon: 0.7268030224906386
% OW    OEOpt     1  0.002843473     3.546255e-06     sigma: 0.32808532768562954, epsilon: 0.5699913501517093
% HS14    OEOpt   1  0               0

% OA     HS14     1  0               0

% OW     HS14     1  0               0
% HS14     HS14   1  8.464e-05       1.5129e-08     sigma: 0.23734081428097492, epsilon: 0.1183807521977659

La siguiente tabla muestran los parámetros Lennard-jones para 2,4-D con el nanotubo del carbono, con agua y consigo misma. La lista (\ref{tab:24Ditp}) es el tipo de átomo dentro de GROMACS para cada átomo de 2,4-D en la figura (\ref{fig:24Dfigure}) descrito en el archivo itp.

\begin{table}[!h]
    \centering
    \begin{tabular}{|c|c|}
    \hline
   CLAro & CL1\\
    CAro &  C4\\
    CAro &  C3\\
      HC &  H2\\
    CAro &  C2\\
   CLAro &  CL\\
    CAro &  C1\\
    CAro &  C6\\
      HC &  H4\\
    CAro &  C5\\
      HC &  H3\\
      OE &  O3\\
    CPos &  C7\\
      HC &  H5\\
      HC &  H6\\
    CPos &  C8\\
   OEOpt &  O2\\
      OA &  O1\\
    HS14 &  H1\\
    \hline
    \end{tabular}
    \caption{lista de átomos en el archivo .itp}
    \label{tab:24Ditp}
\end{table}


\begin{table}[!h]
    \centering
    \begin{tabular}{|c|c|}
    \hline
        $a_1$ $a_2$ & $\sigma$(nm) $\epsilon$(kJ/mol) \\
        \hline
        C    CLAro      &     0.3421967204782815 0.6188115086273643\\
        C     CAro      &     0.3550722828524636 0.29199205084165447\\
        HC	C	        &     0.2915407359639445 0.18121670327032705\\
        C	OE	        &     0.31942690878969837 0.541525315871144\\
        C     CPos      &     0.31730550936068447 0.5332768238073807\\
        C    OEOpt      &     0.34895427699812803 0.37231728284041216\\
        C     HS14      &     0                   0\\
        C	OA	        &     0.31942690878969837 0.541525315871144\\
        OA    CLAro     &     0.3052258307853132 1.2079854835809698\\
        OW    CLAro     &     0.3217319135230157 0.9473561099431141\\
        HS14    CLAro   &     0                   0\\
        CLAro    OEOpt  &     0.33344045848470744 0.830531965485784\\
        CLAro     CAro  &     0.3392864686204393 0.6513496789057335\\
        CLAro     CPos  &     0.3031987401164613 1.1895857035602366\\
        OE    CLAro     &     0.30637119042109173 1.181141361723615\\
        HC    CLAro     &     0.27857941708920847 0.4042418305704644\\
        OE     CAro     &     0.3167103054468882 0.5699996456019336\\
        OA     CAro     &     0.3167103054468882 0.5699996456019336\\
        OW     CAro     &     0.33383744896554585 0.44701914688580746\\
        HC     CAro     &     0.2890612938252534 0.19074538828359391\\
        HS14     CAro   &     0                   0\\
        OEOpt     CAro  &     0.34598655471395084 0.39189436403192596\\
        CAro     CPos   &     0.3146069477063536 0.56131743368072\\
        CAro     CAro   &     0.3520525292750613 0.30734549359078933\\
        HC	OW	        &     0.2741053705029336 0.2774297967529768\\
        HC	OE	        &     0.2600427123263571 0.3537541624242424\\
        HC	OA	        &     0.2600427123263571 0.3537541624242424\\
        HC     HS14     &     0.23734081428097492 0.1183807521977659\\
        HC    OEOpt     &     0.28408067888820254 0.24321827026101842\\
        HC     CPos     &     0.2583156954670584 0.34836585365853656\\
        OE     HS14     &     0                   0\\
        OE    OEOpt     &     0.31125329012564956 0.7268030224906386\\
        OE     CPos     &     0.28302386291723314 1.0410127363636363\\
        OA     CPos     &     0.28302386291723314 1.0410127363636363\\
        OW     CPos     &     0.2983293017984591 0.8164086321626617\\
        HS14     CPos   &     0                   0\\
        OEOpt     CPos  &     0.3091861733013462 0.7157324038901602\\
        OA    OEOpt     &     0.31125329012564956 0.7268030224906386\\
        OW    OEOpt     &     0.32808532768562954 0.5699913501517093\\
        HS14    OEOpt   &     0                   0\\
        OA     HS14     &     0                   0\\
        OW     HS14     &     0                   0\\
        HS14     HS14   &     0.23734081428097492 0.1183807521977659\\
        \hline
    \end{tabular}
    \caption{Tabla de parametros de Lennard-Jones de 2,4-D}
    \label{tab:my_label}
\end{table}


\backmatter  % En esta línea está terminado el trabajo. No quitar
\addcontentsline{toc}{chapter}{Bibliografía} % Agregar Bibliografía al índice
\bibliographystyle{ieeetr}
\bibliography{tesis}      % Para utilizar con BibTeX y con un archivo tesis.bib
%\begin{thebibliography}{XX} % Por si alguien no quiere utilizar BibTeX
%\end{thebibliography}
\end{document}            % Fin del documento. cualquier texto a partir de esta
                          % línea es ignorado por latex.
