%---------------TODO------------
%La introducción deberá contener
%a) motivación principal para estudiar los sistemas estudiando en la tesis
%b) algunos antecedentes de trabajos experimentales para la captacion o arrastre de herbicidas, contaminantes o medicamentos usando nanotubos de carbono y alguna descripcion general sobre nanotubos de carbono y herbicidas o contaminantes
%c) el estudio de estos sistemas desde el punto de vista de la simulación molecular y algunas referencias al respecto
%investigar funcionalizacion de los nanotubos de carbono para tener una respuesta.
% Buscar mas textos e informacion sobre los nanotubos de carbono
%Ideas para presentación. Bien citada todas mis ideas, platicar sobre saneamiento de agua y usos de nanotubos de carbono. 
\chapter{Introducción}

% \textcolor{red}{Reescrbir introduccion, ir de lo general a lo particular, hablar del agua contaminada, despues hablar sobre méxico y la contaminación 2,4-D, hablar sobre los antecedentes de remoción de 24D, hablar sobre nanotubos de carbono y su funcionalización, hablar sobre mec est, dinam. mol., al final hablar sobre la capitulación de la tesis}

En México el 70\% del consumo de agua total se usa en la agricultura \cite{maguey2018}. Adicionalmente, según INECC se han localizado 125 sitios con presencia de herbicidas en agua superficial, subterránea y en suelo \cite{ineec2019}. Además, el incremento de la población genera un incremento de cultivos para alimentación lo cual a su vez genera un crecimiento en el uso de herbicidas o químicos para maximizar la cosecha. Estos compuestos químicos usados terminan en el agua subterránea y superficial.\\

Y es por eso que la ciencia ha desarrollado soluciones, investigando múltiples métodos emergentes de remoción de estos contaminantes, en particular, en la remoción del herbicida 2,4-D estudiado en esta tesis. A continuación se presentan dos ejemplos.

\begin{itemize}
    \item Experimento de remoción del herbicida 2,4-D (ácido 2,4-diclorofenoxiacético) usando un polisacárido no tóxico y biodegradable llamado chitosán \cite{Nunes2019}.
    \item Experimento de remoción del herbicida 2,4-D usando nanotubos de carbono de pared simple (6,5) \cite{rocha2017}.
\end{itemize}

En el caso del segundo experimento es un sistema que puede ser estudiado a mayor detalle \cite{rocha2017} al conducir una simulación computacional para demostrar la adsorción de 2,4-D en el nanotubo de carbono, así como medir propiedades estructurales del sistema y generar información a diferentes condiciones de temperatura del experimento.\\

La nanoestructura usada en esta tesis son los nanotubos de carbono de pared simple (SWCNT). Estos son ampliamente usados en áreas de investigación como ciencias de materiales, nanotecnología, entre otros \cite{KAUR2019} \cite{SARKAR2018}. \\

En el primer capítulo se desarrolla la descripción geométrica y teórica de los SWCNT junto con otros detalles; en el segundo capítulo se presenta la mecánica estadística clásica, se desarrollan los ensambles usados en la simulación, entre otros; en el tercer capítulo se describe la dinámica molecular, en el cuarto capítulo se desarrolla la extracción de propiedades macroscópicas y de estructura; en el quinto capítulo se presentan los resultados y en el sexto capítulo se presentan las conclusiones y perspectivas.

% El saneamiento del agua es un problema que incrementa junto con la población: en 2019 se usó alrededor de 2900$km^3$ de agua para agricultura \cite{un2020} a partir de esto, nuevos métodos y filosofías al respecto han derivado en múltiples innovaciones para atacar el problema como los siguientes: prevención de la contaminación de ríos; plantas de tratamiento de agua de bajo costo y demás.\\ 

% El incremento de la población provoca un incremento de cultivos para alimentación lo cual a su vez genera un crecimiento en el uso de herbicidas o químicos para maximizar la cosecha. Estos químicos usados terminan en el agua subterránea y superficial. En particular, en el río La Laja en Guanajuato se midió en el 2019 con una concentración máxima de 11.309 $\mu g / g$ en suelo y $\mu g / L$ en agua el herbicida 2,4D\cite{ineec2019}.\\

% Las técnicas de remoción de herbicidas del agua son de mucho interés, mucha investigación existe en este campo como por ejemplo:

% \begin{itemize}
%     \item Remoción del herbicida 2,4D usando chitosan, polisacárido no tóxico y biodegradable con excelentes resultados\cite{Nunes2019}.
%     \item Remoción de terbutizalina usando carbón activado mediante carbonización hidrotermal con buenos resultados\cite{tasca2019}.
%     \item Remoción de varios herbicidas usando hongos de tierras contaminadas con herbicidas\cite{Bordjiba2001}.
% \end{itemize}

% Los nanotubos de carbono son estructuras tubulares provenientes de láminas de grafeno. Poseen una geometría característica y propiedades semiconductoras así como conductoras dependiendo de su geometría. Esto las ha hecho un objeto de estudio muy recurrente.

% En los últimos años se ha estudiado mucho sobre las posibles aplicaciones de los nanotubos de carbono en muchas áreas de interés:

% \begin{itemize}
%     \item Por sus propiedades semiconductoras se creó un microprocesador de 32 bits usando nanotubos de carbono por primera vez \cite{Hills2019}.
%     \item Membranas de nanotubos de carbono como candidatos para filtrar agua \cite{IHSANULLAH2019307}.
%     \item En el area de la medicina se ha planteado su uso como nanocápsulas transportadoras de medicamentos \cite{hilder2008}.
% \end{itemize}

% Algunas de estas aplicaciones son perfectas para hacer uso de la simulación molecular por la complejidad que representan estos sistemas. Se han conducido múltiples simulaciones de nanotubos de carbono en la actualidad sobre estas aplicaciones pero por ser un material moderno no se han conducido simulaciones para muchos otros sistemas en los que podría aplicarse. Tal es el caso de la simulación de remoción de herbicidas en el agua usando nanotubos de carbono. En esta tesis se presentan simulaciones de dinámica molecular de un sistema nanotubo de carbono + agua + herbicida 2,4D para verificar la posible remoción de este herbicida usando nanotubos de carbono ya planteado en un artículo \cite{rocha2017}.



